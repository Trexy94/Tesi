% !TEX encoding = UTF-8
% !TEX TS-program = pdflatex
% !TEX root = ../tesi.tex
% !TEX spellcheck = it-IT

%**************************************************************
\chapter{Descrizione dello stage}
\label{cap:descrizione-stage}
%**************************************************************

\intro{In questo capitolosi intende illustrare nel dettaglio le fasi e le considerazioni che hanno portato allo sviluppo del progetto e che sono alla sua base}\\

%**************************************************************
\section{Introduzione al progetto}
\subsection{Considerazioni preliminari}
Prima di cominciare l’approfondimento inerente al progetto di stage, è necessario
esplicitare alcune considerazioni. Innanzitutto, lo stage si è svolto in un’azienda i
cui prodotti software vengono realizzati attraverso l’impiego di procedure, regole e
strumenti idonei a raggiungere la massima qualità. I principi qualitativi fondamentali
su cui Ennova Research si basa sono:
\begin{itemize}
\item Tracciabilità: identificare con certezza lo stato del processo, delle attività e delle
componenti in lavorazione;
\item Identificazione: possibilità di identificare con certezza ogni elemento che entra a
far parte del processo produttivo, rendendo evidenti le modifiche e lo stato di
aggiornamento di ciascun elemento;
\item Riproducibilità: possibilità di ripetere un processo;
\item Documentazione: descrizione delle informazioni necessarie per la realizzazione
del prodotto in modo puntuale e preciso
\item Misurazione: ogni attività deve essere caratterizzata da un insieme di indicatori
che consentano di misurare non solo la qualità, ma anche di tener traccia dei costi (in termini di tempo, nel caso del progetto qui esposto).
\end{itemize}
Questi principi sono stati rigorosamente seguiti nello sviluppo dei plugin al fine di
raggiungere un livello di qualità conforme ai canoni aziendali.
\subsection{Il progetto nello specifico}
Come in precedenza spiegato, l’azienda Ennova Research utilizza la piattaforma Alfresco
per gestire i progetti assegnati dalle aziende clienti e monitorare l’operato dei dipendenti
attraverso l’assegnazione di task e il controllo del loro compimento.
L’azienda ritiene questo tipo di piattaforma fondamentale per il project management
e un’ottima fonte di business. Per questo motivo, l’azienda vuole trasformare la
piattaforma in un prodotto software in grado di svolgere alcune attività senza l’utilizzo
di applicazioni terze. L’ottimizzazione avverrà attraverso l’introduzione di moduli utili
ad automatizzare il più possibile le attività di project management. I moduli svolti
durante il periodo di stage hanno consentito di aggiungere alcune funzioni di tracciamento dei clienti e di gestione dei progetti ad Alfresco.
\subsection{Considerazioni iniziali}
Alfresco, soprattutto nella versione Community, utilizzata dall'azienda perchè Open Source è dotata di pochissime utilità di base, dal momento che il profitto dell'azienda che produce il software proviene dalla consulenza tecnica fornita dai loro esperti e dalla vendita di moduli ulteriori che vanno ad estendere la dotazione di base, moduli disponibili, soprattutto nel caso dei più interessanti, solo tramite la sottoscrizione di un abbonamento, che viene valutato di caso in caso ma è comunque costoso. L'azienda quindi punta a fornire moduli adatti alla gestione di clienti e progetti integrati direttamente in Alfresco, e punta quindi gradualmente ad accentrare in Alfresco più funzionalità possibili, sostituendo gradualmente gli applicativi attualmente utilizzati ad esempio per la gestione dei progetti e la rendicontazione, per poi in un futuro poter proporre a  clienti terzi il prodotto finito.Per questo motivo il progetto di stage si è concretizzato in quanto esposto, cercando quindi di introdurre alcune funzionalità.
\subsubsection{Approccio alla piattaforma}
Le prime due settimane di stage sono state dedicate prevalentemente allo studio della complessa archittetura
e della SDK della piattaforma.
Per affrontare le problematiche appena illustrate è stato effettuato un attento studio
della SDK di Alfresco, individuando il tipo di archetype da utilizzare. La scelta è
ricaduta sull’archtype all-in-one in quanto permette lo sviluppo sia del lato front-end
che di quello back-end, rispecchiando il giusto approccio per i plugin da sviluppare.
Successivamente, è stata effettuata un’attenta attività di analisi e di progettazione per
realizzare quanto dovuto, utilizzando gli strumenti idonei alle diverse fasi.
