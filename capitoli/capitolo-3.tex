% !TEX encoding = UTF-8
% !TEX TS-program = pdflatex
% !TEX root = ../tesi.tex
% !TEX spellcheck = it-IT

%**************************************************************
\chapter{tecnologie e strumenti utilizzati}
\label{cap:tecnologie-strumenti}
%**************************************************************

\intro{In questo capitolo saranno esposte le principali tecnologie utilizzate e i principali strumenti che sono stati utilizzati per portare a compimento il progetto assegnato}\\

%**************************************************************
\section{Tecnologie utilizzate}
Per realizzare quanto è stato richiesto dall'azienda, si sono dovute apprendere svariate nuove tecnologie e linguaggi, a seguito elencati, che vanno ad aggiungersi a quelli pregressi, come ad esempio l'HTML e il CSS, già noti e ampiamente utilizzati nel corso del progetto.
\subsection{Linguaggi di Programmazione}
\subsubsection{Java}
Java è un linguaggio di programmazione orientato agli oggetti a tipizzazione statica,
specificatamente progettato per essere il più possibile indipendente dalla piattaforma
di esecuzione. Uno dei principi fondamentali del linguaggio è espresso dal motto "\textit{write
once, run anywhere}": il codice compilato che viene eseguito su una piattaforma non
deve essere ricompilato per essere eseguito su una piattaforma diversa. Il prodotto
della compilazione è infatti in un formato chiamato bytecode che può essere eseguito da
una qualunque implementazione di un processore virtuale detto Java Virtual Machine.
Alfresco usa Java per il lato backend, soprattutto per i webscript, per i quali è stato largamente utilizzato assieme alla sua API.
\subsubsection{Javascript}
JavaScript è un linguaggio di scripting orientato agli oggetti e agli eventi, comunemente
utilizzato nella programmazione Web lato client per la creazione, in siti web e
applicazioni web, di effetti dinamici interattivi, tramite funzioni di script invocate da
eventi innescati, a loro volta, in vari modi dall’utente sulla pagina web in uso. Tali
funzioni di script possono essere opportunamente inserite in file HTML, in pagine JSP
o in appositi file separati con estensione .js, poi richiamati nella logica di business.
Nel progetto esso è stato usato per popolare dinamicamente le pagine di cui si sono composti i vari moduli e per eseguire i controlli sui form una volta inviate le richieste.
\subsection{Formato per l’interscambio di dati}
\subsubsection{JSON}
JSON (JavaScript Object Notation) è un formato di scambio dati leggero e facile
da leggere e scrivere per le macchine e di non difficile interpretazione per gli umani. Si basa su un sottoinsieme del
linguaggio di programmazione JavaScript, anche se ne è indipendente, ed  inoltre è un formato di testo completamente indipendente da qualsiasi linguaggio.
Per queste caratteristiche è un linguaggio di scambio dati ideale. JSON è costruito su
due strutture:
\begin{itemize}
\item Una collezione di coppie nome/valore, spesso realizzato come un array associativo;
\item Un elenco ordinato di valori, spesso realizzato come una lista.
\end{itemize}
Nel progetto esso è stato usato per poter scambiare i dati tra il lato frontend e il lato backend attraverso il passaggio di oggetti JSON tramite AJAX
\subsection{Tecnica per l’interscambio di dati}
\subsubsection{AJAX}
AJAX, acronimo di Asynchronous Javascript And XML, è una tecnica di sviluppo
software per la realizzazione di applicazioni web asincrone.
Lo sviluppo di applicazioni HTML con AJAX si basa su uno scambio di dati in
background fra web browser e server, che consente l’aggiornamento dinamico di una
pagina web senza esplicito ricaricamento da parte dell’utente.
AJAX è asincrono, nel senso che i dati extra sono richiesti al server e caricati in
background senza interferire con il comportamento della pagina esistente, anche se è possibile, ma sconsigliato, effettuare richieste sincrone. Normalmente
le funzioni richiamate sono scritte con il linguaggio JavaScript. Tuttavia, e a dispetto
del nome, l’uso di JavaScript e di XML non è obbligatorio, come non è detto che le
richieste di caricamento debbano essere necessariamente asincrone.
Nel progetto esso è stato usato per mettere in comunicazione il lato frontend e il lato backend, grazie anche all'utilizzo delle librerie AJAX messe a disposizione dell'API Javascript di Alfresco.
\subsection{Framework}
\subsubsection{Spring}
Spring è un framework per realizzare applicazioni web basate sul Modello MVC
avendo come punti di forza l’inversion of control (tramite dependency injection) e la
aspect oriented programming. Esso si occupa di mappare metodi e classi Java con
determinati url, di gestire differenti tipologie di "viste" restituite al client, di realizzare
applicazioni internazionalizzate e di gestire i cosiddetti temi per personalizzare al
massimo l’esperienza utente. Questo framework è strutturato a livelli, e permette di
scegliere quale dei suoi componenti usare, fornendo nello stesso momento un framework di ottima qualità per lo sviluppo di applicazioni distribuite. Questa architettura a livelli consiste
in diversi moduli (o componenti) ben definiti, ognuno dei quali può rimanere da solo o
essere implementato con altri.
Su questo framework si basa la struttura di base di Alfresco.
\subsubsection{JQuery}
JQuery è un framework nato con il preciso intento di rendere il codice più sintetico
e di limitare al minimo l’estensione degli oggetti globali per ottenere la massima
compatibilità con altre librerie. Grazie a questo principio JQuery è in grado di offrire
un’ampia gamma di funzionalità, che vanno dalla manipolazione degli stili CSS e degli
elementi HTML, agli effetti grafici, per passare a comodi metodi per chiamate AJAX
cross-browser. Nel progetto esso è stato utilizzato per la creazione di popup ed effetti per l'utente.
\subsubsection{Alfresco Surf}
Alfresco Surf è un framework messo a disposizione della piattaforma Alfresco con cui è
possibile creare interfaccia grafiche, modelli e componenti sfruttando script server-side
e template.
Anche se Alfresco stesso sta per abbandonarlo in favore di Angular2, esso è ancora il framework con cui viene distribuito il lato frontend nelle distribuzioni di Alfresco.
\subsection{Protocolli di servizi}
\subsubsection{LDAP}
LDAP è un acronimo che sta per LIGHTWEIGHT DIRECTORY ACCESS
PROTOCOL. Come suggerisce il nome stesso, è un protocollo leggero per accedere
ai servizi di directory, basati sul protocollo X.500. LDAP opera su TCP/IP o su altre
connessioni orientate ai servizi di trasferimento. LDAP nasce per sostituire DAP in
quanto molto oneroso dal punto di vista dell’impiego delle risorse ed è basato
sul modello client-sever: un client LDAP invia una richiesta ad un server LDAP, che
processa la richiesta ricevuta, accede eventualmente ad un directory database e ritorna
dei risultati al client.
Il modello di informazioni di LDAP è basato sulle entry. Un’entry è una collezione di
attributi aventi un unico nome globale: il Distinguished Name (DN). Il DN è usato
per riferirsi ad una particolare entry, senza avere ambiguità.
Ogni attributo dell’entry ha un tipo ed uno o più valori.
In LDAP, le entry di una directory sono strutturate come in una struttura gerarchica
di un albero.\\
L'azienda fa uso di questo protocollo per la profilatura degli utenti in virtù della sua ottima integrazione con Alfresco, e quindi di conseguenza si è dovuta utilizzare questa tecnologia per avere accesso ai dati degli utenti registrati
\subsection{Data Base Management System}
\subsubsection{PostgreSQL}
PostgreSQL  è un completo DBMS ad oggetti rilasciato con licenza libera; spesso viene abbreviato come "Postgres".
PostgreSQL è una reale alternativa sia rispetto ad altri prodotti liberi come ad esempio MySQL, con i quali però comunque condivide molti aspetti del suo linguaggio, che a quelli a codice chiuso come ad esempio Oracle, poiché offre caratteristiche uniche nel suo genere che lo pongono per alcuni aspetti all'avanguardia nel settore dei database. È il DBMS usato in maniera predefinita da Alfresco e per questo motivo è stato scelto per questo progetto.
\section{Strumenti utilizzati}
%\subsubsection{Strumenti di supporto all’attività di Analisi}
%\paragraph{Astah}
%Astah è un software molto diffuso per la produzione di diagrammi UML.
%Le sue potenzialità vanno in realtà oltre alla semplice produzione grafica di diagrammi,
%ma in questo progetto è stato utilizzato esclusivamente per la realizzazione dei
%diagrammi dei casi d’uso inseriti nell’analisi dei requisiti.
\subsection{Strumenti di supporto all’attività di Codifica}
\subsubsection{Eclipse}
Eclipse è un ambiente di sviluppo integrato multi-linguaggio e multi-piattaforma. Ideato da un consorzio di grandi società quali Ericsson, HP, IBM, Intel, MontaVista Software, QNX, SAP e Serena Software, chiamato Eclipse Foundation,
Eclipse è un software libero distribuito sotto i termini della Eclipse Public License ed è lo strumento che viene imposto dall'azienda ai suoi dipendenti, in quanto software libero, quindi la sua adozione per questo progetto è stata scontata.
Questo IDE è stato utilizzato per l’importazione
di un progetto allo scopo di estendere la piattaforma di Alfresco. Le funzionalità
presentate da questo IDE in termini di aiuti al programmatore e la sua integrazione con Maven hanno permesso uno sviluppo rapido ed efficiente dei plugin.
\subsection{Strumenti per il versionamento}
\subsubsection{Git}
Il sistema di versioning adottato per questo progetto e in generale utilizzato dall'azienda è Git, che negli ultimi anni si è affermato come uno
dei migliori sistemi di controllo di versione.
Le caratteristiche per cui si è distinto dagli altri software sono:
\begin{itemize}
\item L’architettura, progettata per essere totalmente distribuita, in modo da
rendere possibile il lavoro e il versionamento offline ed anche un versionamento integrato con i principali IDE;
\item Le funzionalità di branching e merging potenti, rapide e comode, che permettono una efficiente gestione di progetti, anche di grandi dimensioni;
\item Le performance generalmente migliori e la possibilità di pubblicare i repository con i principali protocolli.
\end{itemize}
Ennova Research si appoggia nello specifico su GitLab, un servizio di hosting per progetti basato
su Git.
\subsection{Strumenti per \emph{build automation}}
\subsubsection{Apache Maven}
 Maven è un software usato principalmente per la gestione di progetti Java e \emph{build automation}. Per funzionalità è similare ad Apache Ant, ma basato su concetti differenti. Può essere usato anche in progetti scritti in C\#, Ruby, Scala e altri linguaggi. Il progetto Maven è ospitato da Apache Software Foundation.
Maven usa un costrutto conosciuto come Project Object Model (POM); un file XML che descrive le dipendenze fra il progetto e le varie versioni di librerie necessarie nonché le dipendenze fra di esse. In questo modo si separano le librerie dalla directory di progetto utilizzando questo file descrittivo per definirne le relazioni.
Maven effettua automaticamente il download di librerie Java e plug-in Maven dai vari repository definiti scaricandoli in locale o in un repository centralizzato lato sviluppo. Questo permette di recuperare in modo uniforme i vari file JAR e di poter spostare il progetto indipendentemente da un ambiente all'altro avendo la sicurezza di utilizzare sempre le stesse versioni delle librerie e la possibilità di impostare profili di esecuzione diversi.
Per automatizzare la build e il lancio dell'SDK, Alfresco si basa appunto su Maven.
