% !TEX encoding = UTF-8
% !TEX TS-program = pdflatex
% !TEX root = ../tesi.tex
% !TEX spellcheck = it-IT

%**************************************************************
\chapter{Descrizione dello stage}
\label{cap:descrizione-stage}
%**************************************************************

\intro{In questo capitolosi intende illustrare nel dettaglio le fasi e le considerazioni che hanno portato allo sviluppo del progetto e che sono alla sua base}\\

%**************************************************************
\section{Introduzione al progetto}
\subsection{Considerazioni preliminari}
Prima di cominciare l’approfondimento inerente al progetto di stage, è necessario
esplicitare alcune considerazioni. Innanzitutto, lo stage si è svolto in un’azienda i
cui prodotti software vengono realizzati attraverso l’impiego di procedure, regole e
strumenti idonei a raggiungere la massima qualità. I principi qualitativi fondamentali
su cui Ennova Research si basa sono:
\begin{itemize}
\item Tracciabilità: identificare con certezza lo stato del processo, delle attività e delle
componenti in lavorazione;
\item Identificazione: possibilità di identificare con certezza ogni elemento che entra a
far parte del processo produttivo, rendendo evidenti le modifiche e lo stato di
aggiornamento di ciascun elemento;
\item Riproducibilità: possibilità di ripetere un processo;
\item Documentazione: descrizione delle informazioni necessarie per la realizzazione
del prodotto in modo puntuale e preciso
\item Misurazione: ogni attività deve essere caratterizzata da un insieme di indicatori
che consentano di misurare non solo la qualità, ma anche di tener traccia dei costi (in termini di tempo, nel caso del progetto qui esposto).
\end{itemize}
Questi principi sono stati rigorosamente seguiti nello sviluppo dei plugin al fine di
raggiungere un livello di qualità conforme ai canoni aziendali.
\subsection{Il progetto nello specifico}
Come in precedenza spiegato, l’azienda Ennova Research utilizza la piattaforma Alfresco
per gestire i progetti assegnati dalle aziende clienti e monitorare l’operato dei dipendenti
attraverso l’assegnazione di task e il controllo del loro compimento.
L’azienda ritiene questo tipo di piattaforma fondamentale per il project management
e un’ottima fonte di business. Per questo motivo, l’azienda vuole trasformare la
\section{Il progetto}
piattaforma in un prodotto software in grado di svolgere alcune attività senza l’utilizzo
di applicazioni terze. L’ottimizzazione avverrà attraverso l’introduzione di plugin utili
ad automatizzare il più possibile le attività di project management. I plugin svolti
durante il periodo di stage hanno consentito di superare i limiti della piattaforma
Alfresco relativamente al protocollo di servizio LDAP.
\subsection{Problematiche iniziali}
La piattaforma Alfresco offre numerose funzionalità, una fra queste è la possibilità di
autenticazione degli utenti tramite il protocollo LDAP.
L’azienda ha deciso di utilizzare questo tipo di protocollo per la registrazione delle
informazioni degli utenti, in quanto essendo dotato di una struttura gerarchica ad
albero fornisce alte prestazioni di lettura per una enorme scala di dati, rappresentando
la soluzione migliore per una realtà aziendale.
La piattaforma in questione ha manifestato un grosso limite relativamente al protocollo
LDAP, in quanto gli utenti possono autenticarsi ma non cambiare e recuperare la
propria password tramite il protocollo LDAP.
L’azienda inizialmente per arginare questo problema ha creato le medesime funzionalità
esternamente alla piattaforma perchè all’interno del personale non vi era alcun
dipendente con capacità tecniche tali da poter esser integrate nel sistema; pertanto è
stato chiesto allo stagista Franco Berton di esaminare la documentazione necessaria
per risolvere le problematiche presenti.
\subsubsection{Approccio alla piattaforma}
La prima settimana di stage è stata dedicata prevalentemente allo studio dell’archittetura
e della SDK della piattaforma. Per prendere dimestichezza con il sistema, l’azienda
ha chiesto di cambiare il layout della Login e della Dashboard secondo i mockup creati
appositamente dallo staff grafico.
Queste richieste hanno comportato la creazione di file Javascript per effettuare le
modifiche strutturali delle interfacce servendosi della tecnologia Jquery e la modifica
di alcuni file css per modificare l’aspetto grafico delle interfacce in base ai mockup
presentati.
I cambiamenti strutturali e grafici delle pagine effettuati sono illustrati di seguito:
\subsection{Il progetto nello specifico}
 \subsubsection{Approccio alle problematiche}
Per affrontare le problematiche appena illustrate è stato effettuato un attento studio
della SDK di Alfresco, individuando il tipo di archetype da utilizzare. La scelta è
ricaduta sull’archtype all-in-one in quanto permette lo sviluppo sia del lato front-end
che di quello back-end, rispecchiando il giusto approccio per i plugin da sviluppare.
Successivamente, è stata effettuata un’attenta attività di analisi e di progettazione per
realizzare quanto dovuto, utilizzando gli strumenti idonei alle diverse fasi.
