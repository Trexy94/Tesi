% !TEX encoding = UTF-8
% !TEX TS-program = pdflatex
% !TEX root = ../tesi.tex
% !TEX spellcheck = it-IT

%**************************************************************
%\chapter{Verifica e validazione}
%\label{cap:verifica-validazione}
%**************************************************************
\chapter{Il modulo progetti}
\label{cap:modulo-progetti}
%**************************************************************

\intro{In questo capitolo verrà esposta l'implementazione e le fasi che hanno portato la creazione del modulo progetti}\\

\section{scopo del modulo} il modulo progetti, come quello clienti si pone come scopo quello di creare un semplice sistema di CRUD in un database. Anche in questo caso valgono le considerazioni fatte riguardo al DB e in più è stato richiesto.

Questo modulo si basa fortemente sulle tecniche e le funzioni già implementate e descritte durante la descrizione del precedente modulo, tuttavia si è cercato di eseguire la  maggior parte del lavoro di composizione delle pagine tramite javascript, per testare la velocità della soluzione sia in termine di prestazioni che di tempi di sviluppo. Inoltre è stata inserita una integrazione con LDAP.

Il progetto punta a sfruttare il modulo di inserimento di clienti per aggiungere nuove funzionalità.
A tale scopo, anche il modulo riguardante i clienti andrebbe aggiornato, per sfruttare tutte le funzionalità introdotte. Tuttavia il modulo clienti è completamente funzionante stand alone, mentre il modulo di un nuovo progetto è pensato per funzionare in abbinamento con il modulo della gestione dei clienti e necessita di esso.

\emph{Nota:in seguito si useranno per i nomi le stesse convenzioni già presentate, i percorsi per questo capitolo si omettono, essendo praticamente speculari}

\section{implementazione delle funzionalità}
Aggiunte al DB
Per il corretto funzionamento del modulo sono necessarie le seguenti tabelle aggiuntive:
\begin{lstlisting}
CREATE TABLE progetti
(
  nome text NOT NULL,
  descr text,
  CONSTRAINT progetti_pkey PRIMARY KEY (nome)
)
\end{lstlisting}
Che è una tabella che semplicemente deve contenere la lista dei progetti e la loro descrizione (si assume  che il nome di un progetto sia univoco, altrimenti basta aggiungere un id in auto\_increment alla tabella). La seguente tabella è stata pensata per garantire la possibilità di aggiunta di attributi ad un progetto,  senza aggiungere complessità alla tabella di coloro che sono impiegati in un progetto.
È necessaria per poter gestire il caso di un progetto senza alcun assegnatario.

Più degna di nota è la tabella delle persone impegnate su un progetto:

\begin{lstlisting}
CREATE TABLE lavoratori
(
  mail text NOT NULL,
  project text NOT NULL,
  client serial NOT NULL,
  descr text,
  CONSTRAINT lavoratori_pkey PRIMARY KEY (mail, project, client),
  CONSTRAINT lavoratori_client_fkey FOREIGN KEY (client)
      REFERENCES clienti (id) MATCH SIMPLE
      ON UPDATE NO ACTION ON DELETE NO ACTION
)
\end{lstlisting}

Come si vede vi è qui diretto collegamento tra il progetto e il cliente che lo ha commissionato (si assume nell’assegnazione del progetto che la versione del cliente a cui viene assegnato un progetto sia quella dell’ultima modifica nota eseguita tramite il modulo clienti. Ciò è garantito del fatto di selezionare l’id maggiore, in quanto il tipo serial dell’id cliente non decresce alla cancellazione di un record; si ha quindi la certezza che ad un dato nome di un cliente, la sua ultima versione è quella con id maggiore.

Definizione delle properties

Il sistema fa uso delle seguenti properties, riportate a questo punto per far capire meglio le query utilizzate: 
\begin{lstlisting}
#query configuration
queryselectallprojects=select project,mail,lavoratori.descr,identificativo from lavoratori join clienti on Client=ID order by project
queryselectallclients=select distinct identificativo from clienti
#do not use values in the form paramX! They may be sobstituted!
insertworker=insert into lavoratori values ('value1','value2',(select max(ID) from clienti where identificativo='value3'),'value4');
insertproject= insert into progetti values ('value2','value4')
#server configuration
serverip=localhost
serverid=alfresco
serverpassword=admin
serverport=5432
servername=testdb
servertype=postgresql
serverconnector=jdbc
ldap.authentication.java.naming.provider.url=ldap://192.168.55.73:389
#path in lucene, in the form of PATH:\"/{directory}, the "\"" character is needed 
#because without it it will be escaped by java, java alone puts the final " so 
#you must not put it when you write the query
clientinstallationpath=PATH:\"/app:company_home/st:sites/cm:er/cm:documentLibrary/cm:_x0030_2_x0020_-_x0020_Clients/cm:put_name_here/cm:_x0030_1_x0020_-_x0020_Projects
#down here you can put the names for the folder that are in the client subdirectory
folder.father.project.one=01 - Technical Documentation
folder.father.project.two=02 - Management Documentation
folder.child.one.one=01 - Analysis
folder.child.one.two=02 - Design
folder.child.one.three=03 - Communication and Marketing
folder.child.one.four=04 - Manuals and Tutorials
folder.child.one.five=05 - Reports, Records, Meeting Minutes
folder.child.two.one=01 - Project Management
folder.child.two.two=02 - Financial
folder.child.two.two.one=01 - Proposals
folder.child.two.two.two=02 - Contracts
folder.child.two.two.three=03 - Accounting
folder.child.two.two.four=04 - Invoices
\end{lstlisting}
Come si nota sono configurabili, oltre ai parametri delle varie  connessioni, anche i nomi delle cartelle e le query.

\section{Parte share}
La parte share si basa su una struttura simile a quella del modulo clienti. Per ragioni di tempo, non si è potuto sviluppare un modify per i progetti e la pagina di visualizzazione dei progetti è quindi un semplice dump dei dati, con però la possibilità di inviare una maila  tutti coloro che sono stati assegnati al progetto. La struttura è stata tuttavia predisposta all’aggiunta di funzionalità.

\subsection{Pagina di visualizzazione dei progetti}
Vista la struttura della tabella dei lavoratori, la pagina si occupa di mostrare una riorganizzazione dei dati di quella tabella per dare una migliore leggibilità.
È composta prevalentemente attraverso l’iniezione di codice HTML da parte di funzioni JavaScript che manipolano i risultati del webservice che viene chiamato tramite AJAX.
La parte statica è definita nei file
\begin {itemize}
\item \texttt{/AIOProject-share-amp/src/main/amp/config/alfresco/web-extension/site-webscripts/com/dipendenti/pages/dipendenti.get.desc.xml}
\item \texttt{/AIOProject-share-amp/src/main/amp/config/alfresco/web-extension/site-webscripts/com/dipendenti/pages/dipendenti.get.html.ftl}
\item \texttt{/AIOProject-share-amp/src/main/amp/config/alfresco/web-extension/site-webscripts/com/dipendenti/pages/dipendenti.get.js}
\end{itemize}
Il CSS è invece situato nel file \texttt{/AIOProject-share-amp/src/main/amp/web/css/workers/workers.css}  e il JavaScript nel file \texttt{/AIOProject-share-amp/src/main/amp/web/js/workers/workers.js}. In particolare sono state implementate le seguenti funzioni, oltre a quelle di \texttt{escape/encoding/decoding} di stringhe, che sistemano la codifica delle stringhe:
\begin{itemize}
\item \texttt{setup()}, chiamata al window.onload, che si occupa di fare la chiamata AJAX per ottenere il contenuto della tabella progetti, che è una stringa nella quale il contenuto delle celle è separato dai caratteri  “\#\#\#,” mentre le righe sono separate tramite i caratteri “\#\#\#;” la stringa verrà poi trasformata in codice HTML tramite la funzione
\item \texttt{parse(response)} che si occupa di ritrasformare in un array di array la stringa passata dalla funzione di setup, e di tradurre i dati contenuti in codice HTML  più leggibile e chiaro.
\item \texttt{manda\_mail(destinatari)} che si occupa di generare un piccolo form JQuery per l'inserimento del testo della mail e di fare una chiamata AJAX al servizio SendMails per mandare la mail al gruppo a cui è stato assegnato il progetto;
\end{itemize}

\subsection{Pagina di inserimento di un progetto}
La pagina si occupa di inserire i dati relativi al progetto, a coloro ai quali è assegnato e a creare la cartellatura desiderata assegnando i permessi di visione alla sola cartella tecnica a coloro ai quali è stata assegnato il progetto.
La pagina che si occupa di inserire un progetto e dove è possibile assegnare utenti (viene fornita una lista di quelli presenti nell’LDAP) ad un determinato progetto è definita nella sua parte statica dai file
\begin{itemize}
\item \texttt{/AIOProject-share-amp/src/main/amp/config/alfresco/web-extension/site-webscripts/com/progetti/pages/progetti.get.desc.xml}
\item \texttt{/AIOProject-share-amp/src/main/amp/config/alfresco/web-extension/site-webscripts/com/progetti/pages/progetti.get.html.ftl}
\item \texttt{/AIOProject-share-amp/src/main/amp/config/alfresco/web-extension/site-webscripts/com/progetti/pages/progetti.get.js}
\end{itemize}
Il CSS è invece situato nel file \texttt{/AIOProject-share-amp/src/main/amp/web/css/progetti/progetti.css} e il JavaScript nel file \texttt{/AIOProject-share-amp/src/main/amp/web/js/progetti/progetti.js}.
Il JavaScript utilizza le variabili globali  \texttt{buttons\_unassigned} \texttt{buttons\_assigned assigned}
 per tenere traccia  di quali bottoni sono stati premuti e quindi anche di coloro che sono stati assegnati al progetto. Implementa,  oltre alle funzioni base di manipolazione delle stringhe, anche le seguenti funzioni:
\begin{itemize}
\item \texttt{sendRequest()}, che fa il submit del form al webscript project\_creator che si occupa di eseguire il lato backend dell’inserimento.
\item \texttt{setup()}, che si occupa di fare le richieste necessarie  ad ottenere le informazione necessarie quali la lista dei clienti e la lista degli utenti registrati nell’LDAP.
\item \texttt{process\_response(res)}, che si occupa di processare la lista degli utenti LDAP fornita dalla parte repo e di trasformarla in button html e di settarne l’action onclick.
\item \texttt{refresh()}, che banalmente fa il refresh dei pulsanti mostrati nell’area degli utenti disponibili e in quella degli assegnati.
\item \texttt{assign(mail)}, che assegna un utente alla lista degli utenti a cui è stato assegnato il progetto, si occupa di ricreare il codice HTML dei bottoni e di invocare la funzione refresh(). Implementa anche un sistema per controllare che l’utente che si tenta di assegnare non sia già stato assegnato, poichè testando l’applicativo è capitato di notare che era possibile riuscire ad essere abbastanza rapidi da premere il pulsante di un utente 2 volte prima che la funzione refresh() lo togliesse.
\item \texttt{unassign(mail)}, che si occupa di togliere un utente dalla lista di coloro ai quali è stato assegnato un progetto e di rigenerare il codice html dei pulsanti e di rigenerarli invocando la funzione refresh()
\item \texttt{ControllerTD(),ControllerMD(), ControllerF()}, che banalmente si occupano del check e uncheck e disabilitazione dei figli al cambiamento di uno dei padri nelle checkbox.
\item \texttt{aggiungi()}, che si occupa di gestire il popup per il pulsante di aggiunta di un nuovo utente assegnatario del progetto, invocando poi la funzione \texttt{assign(mail)} per generarne il codice HTML corrispondente al relativo pulsante
\end{itemize}

\subsection{Parte Repo}
In questo modulo, rispetto a quello precedente descritto e come già accennato, si è cercato di ridurre il carico nella parte repo per spostare le computazioni relative alla generazione del codice nel JavaScript.
Tuttavia è stata necessaria l’implementazione di quattro classi Java:
\emph{è importante ricordarsi, se si utilizza la response per ritornare codice HTML o stringhe qualsiasi, di includere il seguente codice per settare la response correttamente}:
\begin{lstlisting}[language=Java]
res.setContentType("text/html; charset=UTF-8");
res.setContentEncoding("UTF-8");
\end{lstlisting}
Altrimenti la codifica dei caratteri non riesce in maniera corretta.
\subsubsection{Clienti.java}
Questa pagina è chiamata senza parametri tramite richiesta POST al webscript “clienti” dalla funzione “setup()” della pagina “progetti”, e si occupa di ritornare il codice HTML necessario a generare un select con i clienti, recuperati dal database, tra le varie option.
Si compone delle seguenti funzioni:
\begin{itemize}
\item \texttt{getValue(String value)}, che recupera una property con una data value
\item \texttt{inizialize\_values ()}, che inizializza le variabili utilizzate, recuperandole dalle properties.
\item \texttt{public void execute(WebScriptRequest req, WebScriptResponse res)} metodo obbligatorio che si può intendere come il metodo che viene invocato alla chiamata Ajax del webscript e che contiene le istruzioni per gestirla.
\item \texttt{execute(Connection con, String query, WebScriptResponse res)}, metodo che esegue la query.
\item \texttt{writedata(ResultSet rs, String result, Connection con)}, metodo che si occupa di ritornare il codice html necessario per generare il select e le option iterando il result set risultato dell’interrogazione del database.
\end{itemize}
\subsubsection{GetUsers.java}
Questa pagina è chiamata senza parametri tramite richiesta POST al webscript “GetUsers” dalla funzione “setup()” della pagina “progetti”, e si occupa di interrogare l’LDAP al fine di ottenere una lista di tutti gli utenti che sono registrati nell’LDAP. La lista è ritornata tramite una stringa con tutti i record separati con una serie di caratteri “sentinella” che fanno da separatori e che poi il Javascript si occupa di rimuovere e di riorganizzare. Esso è composto dalle seguenti funzioni:
\begin{itemize}
\item \texttt{public static DirContext ldapContext()}, che fa da costruttore senza parametri di un ldap context. Viene seguito da
\item \texttt{public static DirContext ldapContext (Hashtable <String,String>env)} che invece è il suo costruttore con parametri.
\item \texttt{public static String getUsers()}, che è il metodo principale della classe e che si occupa di creare la richiesta degli utenti all’LDAP e di iterare la risposta ottenuta al fine di formare una stringa che rispetti le condizioni già citate.
\item \texttt{public static String getcontextFactory()} e \texttt{public static DirContext getLdapcontext()} che  semplicemente ritornano il valore degli omonimi  parametri.
\item \texttt{public void execute(WebScriptRequest req, WebScriptResponse res)}, metodo obbligatorio che si può intendere come il metodo che viene invocato alla chiamata AAJAX del webscript e che contiene le istruzioni per gestirla.
\end{itemize}
\subsubsection{GetWorkers.java}
Questa pagina è chiamata senza parametri tramite richies\_ta POST al webscript “GetWorkers” dalla funzione “setup()” della pagina “dipendenti”. Essa si occupa di ritornare il risultato dell’interrogazione della tabella lavoratori, con il valore delle varie celle separati da un opportuno separatore e le righe separate da un diverso separatore. Ciò permette in seguito al codice JavaScript di ricomporla in un array di array di stringhe  che è possibile iterare.
Esso si compone delle seguenti funzioni:
\begin{itemize}
\item \texttt{getValue(String value)}, che recupera una property con una data value
\item \texttt{inizialize\_values ()}, che inizializza le variabili utilizzate, recuperandole dalle properties.
\item \texttt{public void execute(WebScriptRequest req, WebScriptResponse res)} metodo obbligatorio che si può intendere come il metodo che viene invocato alla chiamata Ajax del webscript e che contiene le istruzioni per gestirla.
\item \texttt{execute(Connection con, String query, WebScriptResponse res)}, metodo che esegue la query.
\item \texttt{writedata(ResultSet rs, String result, Connection con)}, metodo che si occupa di formare la stringa di risposta che corrisponde ai criteri descritti prima.
\end{itemize}
\subsubsection{Project\_creator.java}
Questa pagina è chiamata con svariati parametri tramite richiesta POST al webscript “project\_creator” dalla funzione “sendRequest()” della pagina “progetti”. È la classe più corposa implementata in tutto il progetto. In quanto si occupa di gestire l’inserimento di un nuovo progetto nel DB e di crearne la cartellatura corrispondente, assegnando nel mentre i permessi di accesso e modifica alla sola cartella di documentazione tecnica (e suoi figli) a coloro ai quali è stato assegnato il progetto, in addizione ai gruppi ai quali i permessi vengono invece dati di default.
Essa si compone delle seguenti funzioni:
\begin{itemize}
\item \texttt{private void inizialize\_values()} che inizializza i valori dei vari parametri ottenuti dalle properties.
\item \texttt{public static final String getValue(String value)} che ritorna il valore della property passata come argomento alla funzione.
\item \texttt{public void give\_permissions(String[]users, String permission,NodeRef nodeRef, PermissionService permissionService)} che è usata per dare i permessi all’array di stringhe contenente la lista degli utenti a cui è stato assegnato il progetto.\\
\emph{Dato che è possibile inserire utenti non LDAP e non di Alfresco, può succedere che l’utente a cui si danno i permessi non sia registrato da nessuna parte. Ciò non rompe nulla in quanto Alfresco lo rappresenta nella lista dei permessi con uno spazio bianco fino al momento in cui non viene definito un utente con quello username. In quel momento comparirà il nome. Ciò è stato verificato provandolo effettivamente nell’Alfresco di test. Quindi il seguente caso è tollerato, anche se non è consigliabile ricorrere a questa pratica.}
\item \texttt{public void execute(WebScriptRequest req, WebScriptResponse res)}, che viene invocato alla chiamata AJAX e si occupa di invocare e gestire le varie operazioni compiute.
\item \texttt{private void execute(Connection con, String query)}, che esegue l’inserimento del record nelle tabelle relative agli impiegati e ai progetti.
\item \texttt{private void create\_folder\_tree(String ID, String descr, String[] workers)}, che crea iterativamente l’albero della cartellatura assegnando i permessi di default e invocando \texttt{give\_permissions} alla creazione della cartella relativa alla documentazione tecnica.
\end{itemize}
\subsubsection{SendMails.java}, che si occupa di mandare un messaggio a tutte le mail di una lista, entrambe date in input alla chiamata del servizio. Questa funzione una le API di alfresco, nello specifico il mailAction, che viene istanziato con le seguenti righe:
\begin{lstlisting}[language=Java]
 ActionService actionService = serviceRegistry.getActionService();
 Action mailAction = actionService.createAction(MailActionExecuter.NAME);
\end{lstlisting}
Dal servizio si possono configurare numerosi parametri quali il mittente, il CC, il CCN e praticamente tutti i parametri di una normale mail.
La classe è composta solo dalla funzione \texttt{public void execute(WebScriptRequest req, WebScriptResponse res)}, che è quella chiamata automaticamente quando si richiede il servizio e che si occupa di gestire l'aggiunta dei vari destinatari e dei parametri della mail, oltre ovviamente al suo invio.
\section{lato estetico}
dato che il sistema non è stato ancora dotato delle funzionalità di upload e delete, non si è ritenuto di procedere ad allinearlo con il tema Coral Tree per il momento, quindi non sono stati prodotti mockup. È stato chiesto però di rendere gradevole la presentazione
\subsection{risultati raggiunti}
TODO:immagini