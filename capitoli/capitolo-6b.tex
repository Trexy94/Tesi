% !TEX encoding = UTF-8
% !TEX TS-program = pdflatex
% !TEX root = ../tesi.tex
% !TEX spellcheck = it-IT

%**************************************************************
\chapter{Il modulo tema}
\label{cap:modulo-tema}
%**************************************************************

\intro{In questo capitolo verrà esposta l'implementazione e le fasi che hanno portato la creazione del modulo tema}
\section{Scopo del modulo}
Il modulo tema è molto diverso come scopo e implementazione rispetto ai due precedentemente presentati: infatti questo modulo si basa sull'archetipo Share AMP e si pone come scopo non quello di aggiungere nuove funzionalità, ma quello di realizzare un tema gradevole e moderno, che vada ad aggiungersi a quello già presente nel \gls{KMS} aziendale. La difficoltà maggiore di questo modulo è stata quindi quella di dover far convivere i due temi custom e implementare una soluzione che fosse facilmente manutenibile e che inoltre permettesse anche il cambiamento del tema con la minore difficoltà possibile.\\
Era desiderabile, ma non è stato possibile realizzarlo, che ogni utente potesse scegliere un tema diverso.
\section{Realizzazione}
Nella implementazione dei due temi in una unica AMP è stato sfruttato il fatto, di cui non si era tenuto conto nell’implementazione del tema precedente, che tutti i file di CSS del tema vengono automaticamente caricati assieme alla pagina share, quindi definendo lo stile con classi e ID (anche nuovi e non del tema) nei file principali del tema e non in file specifici per la pagina come nell'implementazione precedente, è possibile così poter cambiare il CSS e le immagini importate e mostrate.
Purtroppo, se si cambia tema, a causa del suo sistema di caching, gli utenti che non hanno effettuato il cambiamento non vedranno sostituito il loro tema fino a un riavvio della componente Share di Alfresco.
\paragraph{}Visto quanto esposto, si è reso necessario un refactoring del modulo tema e dei due moduli, uno che implementa il tema e il login, e l'altro che implementa la funzionalità di recupero password già presenti nel sistema e realizzati precedentemente, per allinearli a quanto prima esposto. Si sono quindi portati tutti i CSS custom nei file del tema, denominato tema Ennova, al fianco del quale si sono implementati i file del nuovo tema, chiamato tema Coral Tree. Si è dovuto quindi togliere dai moduli i riferimenti ai fogli di stile specifici, che altrimenti avrebbero sovrascritto i CSS di tema, e si sono dovuti accorpare i CSS di quei fogli nel tema principale.
\paragraph{} Dato che le modifiche sono avvenute anche riutilizzando il materiale precedente, di seguito si mostrerà solo quanto aggiunto e si trascureranno i file già presenti.
\subsection{File aggiunti nel modulo}
\subsubsection{Diagramma delle cartelle}
Nella figura \ref{fig:cartelle-tema} si illustra in generale quanto è stato prodotto di nuovo per questo modulo. In seguito, nella descrizione di dettaglio, si indicheranno i nomi specifici dei file più importanti che sono stati creati o modificati.
\begin{figure}
\centering
\includegraphics[width=\textwidth]{/diagrammi_cartelle/temifolder.png}
\caption{descrizione generale di quanto creato per il modulo relativo ai temi\label{fig:cartelle-tema}}
\end{figure}
\FloatBarrier
\subsubsection{File di funzionalità}
Per registrare il tema tra quelli di Alfresco stato innanzitutto aggiunto il file CoralTreeTheme.xml, il cui contenuto è riportato nel listato \ref{lst:theme-declaration} che serve a registrare il tema.
\begin{lstlisting}[language=XML, caption=XML della dichiarazione del tema, label=lst:theme-declaration]
<?xml version='1.0' encoding='UTF-8'?>
<theme>
    <title>Coral Tree Theme</title>
    <title-id>theme.CoralTreeTheme</title-id>
    <css-tokens>
        <less-variables>
            <!-- Only for Alfresco 5.x onwards and latest Aikau version -->
        </less-variables>
    </css-tokens>
</theme>
\end{lstlisting}
Successivamente è stato aggiornato il file layout.js, che si occupa di gestire il tema selezionato, e il cui testo integrale è riportato nel listato \ref{code-layout}. Come si può vedere, si controlla nelle costanti di Alfresco quale è il tema attivo e in base a quello si inietta il codice per importare i file corretti.
\begin{lstlisting}[language=JavaScript, caption=codice di layout.js, label=code-layout]
$( window ).load(function() {
    /** CoralTreeTheme **/
    if (Alfresco.constants.THEME == "CoralTreeTheme") {
        document.getElementById("HEADER_SEARCHBOX_FORM_FIELD").setAttribute("placeholder","Chi cerca, trova!");
        var arrayLink = ['/share/res/themes/CoralTreeTheme/dashboard/header/header.css', '/share/res/themes/CoralTreeTheme/dashboard/footer/footer.css'];
        for (i = 0; i < arrayLink.length; i++) {
            var link = document.createElement("link");
            link.rel = 'stylesheet';
            link.type = 'text/css';
            link.href = arrayLink[i];
            document.querySelector("head").appendChild(link);
        }
        var arrayScript = ['/share/res/js/CoralTreeTheme/dashboard/header/header.js', '/share/res/js/CoralTreeTheme/dashboard/footer/footer.js'];
        for (i = 0; i < arrayScript.length; i++) {
            var script = document.createElement("script");
            script.type= "text/javascript";
            script.src = arrayScript[i];
            document.querySelector("head").appendChild(script);
        }
        $(".yui-skin-CoralTreeTheme").css({"display":"block"});
    } else if (Alfresco.constants.THEME == "ennovaTheme") {
        var arrayLink = ['/share/res/themes/ennovaTheme/dashboard/header/header.css', '/share/res/themes/ennovaTheme/dashboard/footer/footer.css'];
        for (i = 0; i < arrayLink.length; i++) {
            var link = document.createElement("link");
            link.rel = 'stylesheet';
            link.type = 'text/css';
            link.href = arrayLink[i];
            document.querySelector("head").appendChild(link);
        }
        var arrayScript = ['/share/res/js/ennovaTheme/dashboard/header/header.js', '/share/res/js/ennovaTheme/dashboard/footer/footer.js'];
        for (i = 0; i < arrayScript.length; i++) {
            var script = document.createElement("script");
            script.type= "text/javascript";
            script.src = arrayScript[i];
            document.querySelector("head").appendChild(script);
        }
        $(".yui-skin-ennovaTheme").css({"display":"block"});
    }
});
\end{lstlisting}
\subsubsection{File di stile}
Per realizzare il tema si sono seguite le linee guida di Alfresco, che dicono di copiare i file di uno dei tema base e di lavorare su quelli, aggiungendo gli stili desiderati. Si hanno quindi:
\begin{itemize}
\item \texttt{footer.css}, che contiene il CSS specifico per il footer
\item \texttt{header.css}, che contiene il CSS specifico per l'header.
\item la cartella \texttt{images}, che contiene le immagini che sono usate dal tema
\item \texttt{presentation.css} che è il CSS principale del tema
\item la cartella \texttt{assets}, che contiene il CSS e i file necessari per i componenti \gls{YUI} utilizzati da Alfresco
\end{itemize}
\section{Aspetto estetico}
Per il lato estetico ci si è ancora una volta dovuti rivolgere al team di grafici che lavorano presso l'azienda per ottenere un mockup. Sono stati utilizzati quelli già prodotti nel passato al momento della presentazione del progetto e della creazione dei primi moduli.
\subsection{Risultati raggiunti}
Nella figura \ref{fig:login} si può vedere l'aspetto della pagina di inserimento di login, mentre nella figura \ref{fig:forgotpwd} si può vedere la pagina per la password dimenticata. Nelle figure \ref{fig:old-home} e \ref{fig:home} si può vedere il confronto tra il tema predefinito di Alfresco e quanto prodotto relativamente alla schermata dell'utente, nelle figure \ref{fig:old-site-home} e \ref{fig:site-home} relativamente alla home page di un sito e nelle figure \ref{fig:old-raccolta} e \ref{fig:raccolta} relativamente alla raccolta dei documenti. Si noti come l'intento sia stato quello di creare un aspetto più moderno con colori più accesi e vivaci rispetto alla normale interfaccia di Alfresco, molto arretrata.
\begin{figure}[!ht]
\centering
\includegraphics[width=\textwidth]{nuove_immagini/login}
\caption{pagina di login\label{fig:login}}
\end{figure}
\begin{figure}[!ht]
\centering
\includegraphics[width=\textwidth]{nuove_immagini/forgotpwd}
\caption{pagina per la password dimenticata\label{fig:forgotpwd}}
\end{figure}
\begin{figure}[!ht]
\centering
\includegraphics[width=\textwidth]{nuove_immagini/old-home}
\caption{home page predefinita di Alfresco \label{fig:old-home}}
\end{figure}
\begin{figure}[!ht]
\centering
\includegraphics[width=\textwidth]{nuove_immagini/home}
\caption{nuovo aspetto della home page \label{fig:home}}
\end{figure}
\begin{figure}[!ht]
\centering
\includegraphics[width=\textwidth]{nuove_immagini/old-site-home}
\caption{pagina principale di un sito di Alfresco, tema predefinito \label{fig:old-site-home}}
\end{figure}
\begin{figure}[!ht]
\centering
\includegraphics[width=\textwidth]{nuove_immagini/site-home}
\caption{pagina principale di un sito di Alfresco, nuovo aspetto \label{fig:site-home}}
\end{figure}
\begin{figure}[!ht]
\centering
\includegraphics[width=\textwidth]{nuove_immagini/old-raccolta}
\caption{raccolta documenti predefinita di Alfresco \label{fig:old-raccolta}}
\end{figure}
\begin{figure}[!ht]
\centering
\includegraphics[width=\textwidth]{nuove_immagini/raccolta}
\caption{nuovo aspetto della raccolta documenti \label{fig:raccolta}}
\end{figure}