% !TEX encoding = UTF-8
% !TEX TS-program = pdflatex
% !TEX root = ../tesi.tex
% !TEX spellcheck = it-IT

%**************************************************************
\chapter{Il modulo tema}
\label{cap:modulo-tema}
%**************************************************************

\intro{In questo capitolo verrà esposta l'implementazione e le fasi che hanno portato la creazione del modulo tema}
\section{scopo del modulo}
Il modulo tema è molto diverso come scopo e implementazione rispetto ai due precedentemente presentati: infatti questo modulo si basa sull'archetipo Share AMP e si come scopo non quello di aggiungere nuove funzionalità, ma quello di realizzare un tema gradevole e moderno, che vada ad aggiungersi a quello già presente nel KMS aziendale. La difficoltà maggiore di questo modulo è stata quindi quella di dover far convivere i due temi custom e implementare una soluzione che fosse facilmente manutenibile e permettesse anche il cambiamento del tema con la difficoltà minore possibile.
Era desiderabile, ma non è stato possibile, che ogni utente potesse scegliere un tema diverso.
\section{Realizzazione}
Nella implementazione dei due temi in una unica AMP è stato sfruttato il fatto, che nell'implementazione del tema precedente non era stato tenuto conto, che tutti i file di CSS del tema vengono automaticamente caricati assieme alla pagina share, quindi definendo lo stile con classi e ID (anche nuovi e non del tema) nei file principali del tema e non in file specifici per la pagina come nell'implementazione precedente, è possibile poter cambiare il CSS e le immagini importate e mostrate.
Purtroppo, se si cambia tema, a causa del suo sistema di caching, gli utenti che non hanno effettuato il cambiamento non vedranno sostituito il loro tema fino a un riavvio della componente Share di Alfresco.
\paragraph{}visto quanto esposto, si è reso necessario un refactoring del modulo tema e dei due moduli, uno che implementa il tema e il login, e l'altro che implementa la funzionalità di recupero password già presenti nel sistema e realizzati precedentemente, per allinearli a quanto prima esposto. Si sono quindi portati tutti i css custom nei fai del tema, denominato tema Ennova, al fianco del quale si sono implementati i file del nuovo tema, chiamato tema Coral Tree. Si è dovuto quindi togliere dai moduli i riferimenti ai fogli di stile specifici, che altrimenti avrebbero sovrascritto i CSS di tema, e si sono dovuti accorpare i CSS di quei fogli nel tema principale.
\paragraph{} Siccome le modifiche son avvenute anche riutilizzando il materiale precedente, a seguito si mostrerà solo quanto aggiunto e si trascureranno i file già presenti.
\subsection{File aggiunti nel modulo}
\subsubsection{File di funzionalità}
\subsubsection{File di stile}
\section{Lato estetico}
Per il lato estetico ci si è ancora una volta dovuti rivolgere al team di grafici che lavorano presso l'azienda per ottenere un mockup. Sono stati utilizzati quelli già prodotti nel passato al momento della presentazione del progetto e della creazione dei primi moduli.
\subsection{Risultati ottenuti}
TODO:immagini