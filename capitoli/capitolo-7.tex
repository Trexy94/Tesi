% !TEX encoding = UTF-8
% !TEX TS-program = pdflatex
% !TEX root = ../tesi.tex
% !TEX spellcheck = it-IT

%**************************************************************
%**************************************************************
%
%**************************************************************
%\section{Consuntivo finale}
%
%**************************************************************
%\section{Raggiungimento degli obiettivi}
%
%**************************************************************
%\section{Conoscenze acquisite}
%
%**************************************************************
%\section{Valutazione personale}
\chapter{Conclusioni}
\label{cap:conclusioni}

Il tirocinio formativo si è svolto secondo i tempi inizialmente concordati,
 non ci sono stati ritardi evidenti e i risultati ottenuti hanno pienamente
soddisfatto le aspettative dell’azienda, dimostrando l'effettiva possibilità di costruire in 
Alfresco una piattaforma evoluta e complessa per la gestione dei progetti e dei clienti. È stata
espressa infatti da parte dell’azienda la volontà di proseguire allo sviluppo e all’integrazione
del sistema in produzione e sono previste ulteriori espansioni del sistema in futuro, per fornire un sistema completo e vendibile come scopo ultimo.
\section{Prospettive future}
Il sistema, ora in produzione, punta ad essere espanso con nuove funzionalità per la gestione dei progetti e dei loro processi, al fine di garantire una gestione migliore degli stessi e dei loro costi, oltre alla volontà di integrare in Alfresco elementi social.
La piattaforma dei clienti in futuro verrà integrata con i progetti ad essi relativi e, come fine ultimo, verrà dotata di ulteriori funzionalità allo stato attuale non concretamente possibili, come ad esempio la creazione di fatture intelligenti che tengano conto di quanto rendicontato in un ipotetico futuro modulo progetti più evoluto.
\section{Difficoltà e limiti riscontrati}
Il progetto è stato svolto senza particolari intoppi, tuttavia vi sono state nel progetto alcune difficoltà, sia di natura tecnica che dovute ad altri fattori, di seguito riportate:
\begin{itemize}
\item la completa inesperienza sul sistema Alfresco, sul suo funzionamento e sulle tecnologie che utilizza, che, anche se mitigata dall'aiuto ricevuto, ha comportato tempi di apprendimento lunghi, data anche la sua complessa architettura.
Ciò ha portato infatti a scelte di cui non sono completamente soddisfatto per quanto riguarda l'implementazione di alcune funzionalità, che, con l'esperienza acquisita anche solo alla fine dello stage, sarebbe stata molto diversa;
\item i tempi piuttosto lunghi della compilazione e lancio dell'All-in-one SDK, dato che il codice Java non gode del RAD, quindi ogni aggiunta, prima di poter essere provata nell'SDK, doveva essere preceduta dal rilancio dell'SDK, che nel dispositivo fornito dall'azienda poteva anche durare più di 30 minuti;
\item l'iniziale assenza di una piattaforma dove poter effettivamente provare i moduli prima di portarli nell'ambiente di produzione, che è stata opportunamente creata sempre nell'ambito dello stage.
\end{itemize}
\section{Considerazioni personali}
Dal punto di vista formativo l'esperienza è stata sicuramente soddisfacente, in quanto mi ha permesso di avere un contatto con un'azienda che opera nel settore informatico ed avere esperienza diretta di come si lavora in quest'ambiente; sono restato molto soddisfatto del clima e dell'accoglienza che mi è stata riservata nell'azienda, oltre all'opportunità che mi è stata da loro concessa di poter continuare a lavorare presso di loro.
È stata una esperienza estremamente interessante e degna conclusione del mio percorso di studi.