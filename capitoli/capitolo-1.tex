% !TEX encoding = UTF-8
% !TEX TS-program = pdflatex
% !TEX root = ../tesi.tex
% !TEX spellcheck = it-IT

%**************************************************************
\chapter{Introduzione}
\label{cap:introduzione}
%**************************************************************

In questo capitolo verrà brevemente esposto il contesto in cui si è svolto lo stage, descrivendo le motivazioni che hanno spinto l'azienda a proporre questo stage \\

\noindent Esempio di utilizzo di un termine nel glossario \\
\gls{api}. \\

\noindent Esempio di citazione in linea \\
\cite{site:agile-manifesto}. \\

\noindent Esempio di citazione nel pie' di pagina \\
citazione\footcite{womak:lean-thinking} \\

%**************************************************************
\section{L'azienda}

Ennova Research srl è un’azienda che opera nel settore ICT e realizza soluzioni
informatiche altamente tecnologiche ed affidabili, che le permettono di agire con
successo in settori di business come quello della Pubblica Amministrazione, delle grandi
Corporate Bancarie, delle Multinazionali ICT e della Grande Distribuzione.
E’ partner di grandi attori del mercato nazionale e internazionale quali Engineering,
Toshiba, EMC, HP, Novell, Nvidia, etc. .
Ennova Research si distingue nel campo delle tecnologie open source utilizzate per la
realizzazione di soluzioni multimediali avanzate destinate ai mercati B2C e B2B e di
applicativi software destinati al settore del mobile.

%**************************************************************
\section{L'idea}

Introduzione all'idea dello stage.

%**************************************************************
\section{Organizzazione del testo}

\begin{description}
    \item[{\hyperref[cap:processi-metodologie]{Il secondo capitolo}}] descrive ...
    
    \item[{\hyperref[cap:descrizione-stage]{Il terzo capitolo}}] approfondisce ...
    
    \item[{\hyperref[cap:analisi-requisiti]{Il quarto capitolo}}] approfondisce ...
    
    \item[{\hyperref[cap:progettazione-codifica]{Il quinto capitolo}}] approfondisce ...
    
    \item[{\hyperref[cap:verifica-validazione]{Il sesto capitolo}}] approfondisce ...
    
    \item[{\hyperref[cap:conclusioni]{Nel settimo capitolo}}] descrive ...
\end{description}

Riguardo la stesura del testo, relativamente al documento sono state adottate le seguenti convenzioni tipografiche:
\begin{itemize}
	\item gli acronimi, le abbreviazioni e i termini ambigui o di uso non comune menzionati vengono definiti nel glossario, situato alla fine del presente documento;
	\item per la prima occorrenza dei termini riportati nel glossario viene utilizzata la seguente nomenclatura: \emph{parola}\glsfirstoccur;
	\item i termini in lingua straniera o facenti parti del gergo tecnico sono evidenziati con il carattere \emph{corsivo}.
\end{itemize}