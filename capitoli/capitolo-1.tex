% !TEX encoding = UTF-8
% !TEX TS-program = pdflatex
% !TEX root = ../tesi.tex
% !TEX spellcheck = it-IT

%**************************************************************
\chapter{Introduzione}
\label{cap:introduzione}
%**************************************************************

\intro{In questo capitolo verrà brevemente esposto il contesto in cui si è svolto lo stage, descrivendo le motivazioni che hanno spinto l'azienda a proporre questo stage}\\

%\noindent Esempio di utilizzo di un termine nel glossario \\
%\gls{api}. \\
%
%\noindent Esempio di citazione in linea \\
%\cite{site:agile-manifesto}. \\
%
%\noindent Esempio di citazione nel pie' di pagina \\
%citazione\footcite{womak:lean-thinking} \\

%**************************************************************
\section{L'azienda}

Ennova Research SRL è un’azienda che opera nel settore \gls{ICT} e realizza soluzioni
informatiche altamente tecnologiche ed affidabili, che le permettono di agire con
successo in settori di business come quello della Pubblica Amministrazione, delle grandi
Corporazioni Bancarie, delle Multinazionali \gls{ICT} e della Grande Distribuzione.
È partner di grandi attori del mercato nazionale e internazionale quali Engineering,
Toshiba, EMC, HP, Novell, Nvidia, ed altri.
Ennova Research si distingue anche nel campo delle tecnologie open source utilizzate per la
realizzazione di soluzioni multimediali avanzate destinate ai mercati \gls{B2C} e \gls{B2B} e di
applicativi software destinati al settore del mobile, ad esempio Slash, che sfrutta le \gls{API} di Twitter.
L'azienda inoltre investe molto in ricerca e sviluppo ed è sempre pronta ad esplorare nuove tecnologie.

%**************************************************************
\section{Lo stage}

Il progetto di stage svoltosi all’interno dell’azienda Ennova Research è principalmente consistito nello sviluppo di moduli per Alfresco, che è il \gls{KMS} che l'azienda ha incominciato da relativamente poco ad utilizzare e a cui intende sviluppare alcune funzionalità, quali ad esempio la gestione dei clienti e dei progetti, che al momento è svolta con l'ausilio di altri applicativi, e la rendicontazione, svolta anch'essa mediante l'utilizzo di altri applicativi.
L'azienda tuttavia si pone l'obiettivo di creare non solo un prodotto necessario ai propri bisogni interni, ma anche un prodotto vendibile e che generi profitto attraverso la vendita dello stesso ad aziende clienti. Il progetto è stato denominato Coral Tree e lo stage si è posto l'obiettivo di iniziare a porre le basi per questo progetto, attraverso la creazione di un modulo che introduca il nuovo tema per dare l'aspetto desiderato al prodotto e la creazione dei primi moduli che introducano le funzionalità precedentemente citate.
%**************************************************************

\section{Organizzazione del testo}
\begin{description}
    \item[{\hyperref[cap:descrizione-stage]{Il secondo capitolo}}] descrive brevemente le condizioni e le metodologie utilizzate per lo stage.
    \item[{\hyperref[cap:tecnologie-strumenti]{Il terzo capitolo}}] descrive le tecnologie e gli strumenti utilizzati durante lo stage.    
		 \item[{\hyperref[cap:architettura]{Il quarto  capitolo}}] descrive brevemente come creare un modulo e l'architettura di Alfresco.
		\item i tre capitoli successivi descriveranno in maniera separata i vari moduli sviluppati, dato che il modulo che aggiunge il nuovo tema differisce
		sostanzialmente nella sua realizzazione dai moduli che aggiungono funzionalità nel senso stretto.
		Si hanno quindi i capitoli
		\begin{description}
		\item[{\hyperref[cap:modulo-tema]{il modulo tema}}];
		\item[{\hyperref[cap:modulo-clienti]{il modulo clienti}}];
		\item[{\hyperref[cap:modulo-progetti]{ il modulo progetti}}];
		\end{description}		
		Alla fine verrà presentata una considerazione finale nel capitolo \hyperref[cap:conclusioni]{"Conclusioni"}
\end{description}