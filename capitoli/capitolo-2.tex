% !TEX encoding = UTF-8
% !TEX TS-program = pdflatex
% !TEX root = ../tesi.tex
% !TEX spellcheck = it-IT

%**************************************************************
\chapter{Descrizione dello stage}
\label{cap:descrizione-stage}
%**************************************************************

\intro{In questo capitolo si intende illustrare nel dettaglio le fasi e le considerazioni che hanno portato allo sviluppo del progetto e che sono alla base della proposta di stage}\\

%**************************************************************
\section{Introduzione al progetto}
\subsection{Considerazioni preliminari}
Per prima cosa, è necessario premettere che l'azienda in cui si è svolto lo stage per i suoi prodotti software richiede l'impiego di procedure, regole e strumenti idonei a raggiungere il miglior risultato possibile.
In particolare, durante lo stage sono stati richiesti i seguenti punti:
\begin{itemize}
\item Tracciabilità: è stato spesso richiesto di rendere conto dello stato di avanzamento dei processi, delle componenti e delle attività in lavorazione, al fine di verificare il rispetto della tabella di marcia,
\item Documentazione: descrizione di quanto è stato fatto, delle strategie utilizzate e dei dettagli implementativi in modo puntuale e preciso, per far sì che quanto è stato fatto generi utilità all'azienda e sia riutilizzabile e replicabile in futuro, senza dover rifare tutte le attività, avendo già una strategia e una soluzione pronte e immediatamente utilizzabili.
\item Misurazione: per ogni attività eseguita, deve essere misurabile il suo costo(in termini di tempo, nel caso del progetto qui esposto), e la sua qualità complessiva, in base a vari indicatori.
\end{itemize}
Quanto esposto è stato tenuto in grande considerazione durante lo sviluppo, al fine di realizzare un prodotto quanto più possibile conforme agli standard di qualità aziendali
\subsection{Considerazioni iniziali}
Alfresco, soprattutto nella versione Community, utilizzata dall'azienda perché Open Source è dotata di pochissime utilità di base, dal momento che il profitto dell'azienda che produce il software proviene dalla consulenza tecnica fornita dai loro esperti e dalla vendita di moduli ulteriori che vanno ad estendere la dotazione di base, moduli disponibili, soprattutto nel caso dei più interessanti, solo tramite la sottoscrizione di un abbonamento, che viene valutato di caso in caso ma è comunque costoso. L'azienda quindi punta a fornire moduli adatti alla gestione di clienti e progetti integrati direttamente in Alfresco, e punta quindi gradualmente ad accentrare in Alfresco più funzionalità possibili, sostituendo gradualmente gli applicativi attualmente utilizzati ad esempio per la gestione dei progetti e la rendicontazione, per poi in un futuro poter proporre a  clienti terzi il prodotto finito. Per questo motivo il progetto di stage si è concretizzato in quanto esposto, cercando quindi di introdurre alcune funzionalità.
\subsection{Approccio alla piattaforma}
Le prime due settimane di stage sono state dedicate quasi interamente allo studio della complessa architettura
e della SDK della piattaforma.
Per affrontare quanto appena illustrate è stato effettuato un attento studio
della SDK di Alfresco, scegliendo il tipo di archetipo da utilizzare. La scelta è
ricaduta sull’archetipo all-in-one in quanto permette lo sviluppo sia del lato front-end
che di quello back-end in simultanea, senza dover distribuire ed installare separatamente un modulo per ogni parte sviluppata.
In seguito, si sono analizzate le richieste poste dall'azienda in merito alle funzionalità desiderate al fine di individuare le strategie e i metodi migliori per realizzare quanto richiesto
\subsection{Ambiente di test}
Assieme allo sviluppo dei vari moduli, è proceduta a pari passo anche la creazione e il mantenimento di un ambiente di test dove poter testare i moduli in maniera sicura prima di portarli in produzione e applicare i moduli nel \gls{KMS} aziendale. Ciò è stato espressamente richiesto dall'azienda e quindi i moduli prodotti, non appena la loro installazione era possibile, sono stati caricati e provati nel \gls{KMS} di test che replicava quello dell'azienda, e successivamente, prima del rilascio finale, in una copia esatta del software del \gls{KMS} di produzione.
\section{Vincoli}
per quanto riguarda l'implementazione, non sono stati imposti allo stagista vincoli stringenti per quanto riguarda i dettagli implementativi e accessibilità, anche se è stato esplicitato che il codice doveva essere facile da manutenere e che tutti i parametri e le stringhe utilizzate dovevano essere collocate in opportuni file di properties, in maniera da rendere facile configurazione e localizzazione.