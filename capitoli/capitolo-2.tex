% !TEX encoding = UTF-8
% !TEX TS-program = pdflatex
% !TEX root = ../tesi.tex
% !TEX spellcheck = it-IT

%**************************************************************
\chapter{Descrizione dello stage}
\label{cap:descrizione-stage}
%**************************************************************

\intro{In questo capitolo si intende illustrare nel dettaglio le fasi e le considerazioni che hanno portato allo sviluppo del progetto e che sono alla base della proposta di stage}\\

%**************************************************************
\section{Introduzione al progetto}
\subsection{Considerazioni preliminari}
Per prima cosa, è necessario premettere che l'azienda in cui si è svolto lo stage per i suoi prodotti software richiede l'impiego di procedure, regole e strumenti idonei a raggiungere il miglior risultato possibile.
In particolare, durante lo stage sono stati richiesti i seguenti punti:
\begin{itemize}
\item Tracciabilità: è stato spesso richiesto di rendere conto dello stato di avanzamento dei processi, delle componenti e delle attività in lavorazione, al fine di verificare il rispetto della tabella di marcia,
\item Documentazione: descrizione di quanto è stato fatto, delle strategie utilizzate e dei dettagli implementativi in modo puntuale e preciso, per far sì che quanto fatto generi utilità all'azienda e sia riutilizzabile e replicabile in futuro, senza dover rifare tutte le attività, avendo già una strategia e una soluzione pronte e immediatamente applicabili.
\item Misurazione: per ogni attività eseguita, deve essere misurabile il suo costo(in termini di tempo, nel caso del progetto qui esposto) e la sua qualità complessiva, in base a vari indicatori.
\end{itemize}
Quanto esposto è stato tenuto in grande considerazione durante lo sviluppo, al fine di realizzare un prodotto quanto più possibile conforme agli standard di qualità aziendali
\subsection{Considerazioni iniziali}
Alfresco, soprattutto nella versione Community, utilizzata dall'azienda perché Open Source, è dotata di pochissime utilità di base, dal momento che il profitto dell'azienda che produce il software proviene dalla consulenza tecnica fornita dai loro esperti e dalla vendita di moduli ulteriori che vanno ad estendere la dotazione di base, moduli disponibili, soprattutto nel caso dei più interessanti, solo tramite la sottoscrizione di un abbonamento, che viene valutato di caso in caso ma è comunque costoso. L'azienda quindi punta a fornire moduli adatti alla gestione di clienti e progetti integrati direttamente in Alfresco, e punta quindi gradualmente ad accentrare in Alfresco più funzionalità possibili, sostituendo gradualmente gli applicativi attualmente utilizzati ad esempio per la gestione dei progetti e la rendicontazione, per poi in un futuro poter proporre a  clienti terzi il prodotto finito. Per questo motivo il progetto di stage si è concretizzato in quanto esposto, cercando quindi di introdurre alcune funzionalità.
\subsection{Approccio alla piattaforma}
Le prime due settimane di stage sono state dedicate quasi interamente allo studio della complessa architettura
e della SDK della piattaforma.
Per affrontare quanto appena evidenziato è stato effettuato un attento studio
della SDK di Alfresco, scegliendo il tipo di archetipo da utilizzare. La scelta è
ricaduta sull’archetipo all-in-one in quanto permette lo sviluppo sia del lato front-end
che di quello back-end in simultanea, senza dover distribuire ed installare separatamente un modulo per ogni parte sviluppata.
In seguito, si sono analizzate le richieste poste dall'azienda in merito alle funzionalità desiderate al fine di individuare le strategie e i metodi migliori per realizzare quanto richiesto
\subsection{Ambiente di test}
Assieme allo sviluppo dei vari moduli, su espressa richiesta dell’azienda, è proceduta a pari passo anche la creazione e il mantenimento di un ambiente di test dove poter testare i moduli in maniera sicura prima di portarli in produzione e applicarli nel \gls{KMS} aziendale. Nello specifico quindi i moduli prodotti, non appena la loro installazione era possibile, sono stati caricati e provati nel \gls{KMS} di test che replicava quello dell'azienda, e successivamente prima del rilascio finale, in una copia esatta del software del \gls{KMS} di produzione.
\section{Vincoli}
Per quanto riguarda l'implementazione, non sono stati imposti allo stagista vincoli stringenti per quanto riguarda i dettagli implementativi e l'accessibilità, anche se è stato esplicitato che il codice doveva essere facile da manutenere e che tutti i parametri e le stringhe utilizzate dovevano essere collocate in opportuni file di properties, in maniera da rendere facile configurazione e localizzazione. Non sono stati specificati requisiti di qualità, in quanto è stato indicato di rispettare gli standard aziendali.
\section{Metodologia di sviluppo}
Per lo sviluppo si è ricorsi al framework agile Scrum, che è quello utilizzato dall'azienda, eseguendo ovviamente daily Scrum quotidianamente per fare il punto sulla situazione e facendo sprint di durata di una/due settimane.
\section{Requisiti funzionali}
Tenuto conto dei vincoli elencati nei punti precedenti, per prima cosa si è eseguita un'analisi relativa ai requisiti funzionali.
Per rendere più chiare le priorità nello sviluppo è stata adottata una simbologia di
classificazione dei requisiti.
Ad ogni Requisito Funzionale è associato un codice del tipo R.X.Y dove:
\begin{itemize}
\item R è l’abbreviazione di Requisito
\item X è una lettera tra:
\begin{itemize}
	\item O per Requisito Obbligatorio
	\item F per Requisito Facoltativo
	\item D per Requisito Desiderabile
\end{itemize}
\item Y è il numero identificativo del Requisito
\end{itemize}
Anche se la realizzazione ha comportato la realizzazione di tre moduli diversi, nella tabella \ref{tab:requisiti-funzionali} i requisiti verranno presentati assieme e nella loro forma completa.
\begingroup
\renewcommand\arraystretch{2}
\begin{longtable}{p{.20\textwidth} p{.60\textwidth} p{.20\textwidth}}
\hline\hline
\textbf{Requisito} & \textbf{Descrizione} & \textbf{Stato}\\
\hline
\hline
R.O.1    & L'applicazione deve permettere il CRUD dei clienti & Completato \\
\hline
R.O.1.1  & Deve essere possibile creare un cliente & Completato \\
\hline
R.O.1.1.1  & Deve essere possibile assegnare un nome ad un cliente & Completato \\
\hline
R.O.1.1.2  & Deve essere possibile assegnare la partita IVA o il codice fiscale ad un cliente & Completato \\
\hline
R.O.1.1.3  & Deve essere possibile assegnare una descrizione ad un cliente & Completato \\
\hline
R.O.1.1.4  & Deve essere possibile assegnare la data di inizio del rapporto con un cliente & Completato \\
\hline
R.O.1.1.5  & Deve essere possibile assegnare la data di fine del rapporto con un cliente & Completato \\
\hline
R.O.1.1.6  & Deve essere possibile assegnare un identificativo ad un cliente & Completato \\
\hline
R.O.1.1.7  & Deve essere possibile creare una cartella per un cliente & Completato \\
\hline
R.O.1.1.7.1  & Le cartelle dei clienti devono avere lo stesso nome definito nell'identificativo & Completato \\
\hline
R.O.1.1.7.2  & Le cartelle dei clienti devono avere la descrizione inserita in fase di creazione del cliente & Completato \\
\hline
R.O.1.1.7.3  & Le cartelle dei clienti devono avere gli opportuni permessi di visibilità & Completato \\
\hline
R.O.1.1.7.4  & Deve essere possibile scegliere quali sottocartelle si vogliono creare per un cliente & Completato \\
\hline
R.O.1.2  & Deve essere possibile visualizzare i dati di un cliente & Completato \\
\hline
R.O.1.2.1  & Deve essere possibile raggiungere la cartella relativa ad un cliente & Completato \\
\hline
R.O.1.3  & Deve essere possibile modificare i dati di un cliente & Completato \\
\hline
R.O.1.3.1  & Gli aggiornamenti di un cliente non devono essere retroattivi su eventuali attributi ad essi assegnati & Completato \\
\hline
R.O.1.3.2  & Devono essere mantenuti sia i dati precedenti alla modifica sia quelli aggiornati & Completato \\
\hline
R.O.1.4  & Deve essere possibile cancellare i dati di un cliente & Completato \\
\hline
R.O.1.4.1  & La cancellazione di un cliente deve essere opportunamente preceduta da un messaggio di allarme & Completato \\
\hline
R.O.1.5  & Si deve poter accedere all'inserimento di un cliente tramite un'aggiunta alla voce delle azioni possibili & Completato \\
\hline
R.O.1.6    & Il CRUD dei clienti deve essere possibile solo per gli amministratori & Completato \\
\hline
R.D.2    & L'applicazione deve permettere il CRUD dei progetti & Non completato \\
\hline
R.D.2.1    & Deve essere possibile creare un progetto & Completato \\
\hline
R.D.2.1.1    & Deve essere possibile assegnare un nome ad un progetto & Completato \\
\hline
R.D.2.1.2    & Deve essere possibile assegnare una descrizione ad un progetto & Completato \\
\hline
R.D.2.1.3    & Deve essere possibile assegnare dei lavoratori ad un progetto & Completato \\
\hline
R.F.2.1.3.1    & Deve essere possibile ottenere i lavoratori dalla piattaforma LDAP & Completato \\
\hline
R.F.2.1.3.2   & Deve essere possibile mandare una mail a coloro ai quali è stato assegnato un determinato progetto & Completato \\
\hline
R.F.2.1.3.2.1   & Deve essere possibile scrivere il testo della mail & Completato \\
\hline
R.D.2.1.4  & La creazione di un progetto deve generare anche le relative cartelle & Completato \\
\hline
R.D.2.1.4.1  & Deve essere possibile specificare quali sottocartelle si vogliono creare & Completato \\
\hline
R.F.2.1.4.2  & Le cartelle devono essere visibili solo agli amministratori e a coloro ai quali è stato assegnato un progetto & Completato \\
\hline
R.D.2.1.5  & Deve essere possibile assegnare un cliente un progetto & Completato \\
\hline
R.D.2.2    & Deve essere possibile visualizzare i dati di un progetto &Completato \\
\hline
R.F.2.3    & Deve essere possibile modificare i dati di un progetto & Non completato \\
\hline
R.F.2.4    & Deve essere possibile cancellare un progetto & Non completato \\
\hline
R.D.2.5    & Il CRUD dei progetti deve essere possibile solo per gli amministratori & Completato \\
\hline
R.O.3  & Deve essere creato un modulo per il nuovo tema & Completato \\
\hline
R.O.3.1  & Il nuovo tema deve poter convivere con i moduli già esistenti in azienda & Completato \\
\hline
R.O.3.2  & Deve essere possibile cambiare il tema & Completato \\
\hline
R.O.3.3  & Il tema deve essere il più simile possibile a quello specificato nel mockup & Completato \\
\hline
R.F.3.4  & Deve essere possibile per ogni utente poter scegliere il proprio tema & Non Completato \\
\hline
R.O.4    & Deve essere possibile configurare i parametri utilizzati dai moduli & Completato \\
\hline
R.O.4.1    & Deve essere possibile configurare le query utilizzate & Completato \\
\hline
R.O.4.2    & Deve essere possibile configurare i nomi delle cartelle create & Completato \\
\hline
\caption{Tabella del tracciamento dei requisiti funzionali}
\label{tab:requisiti-funzionali}
\end{longtable}
\endgroup

Per riassumere, sono stati realizzati:
\begin{itemize}
\item il 100\% dei requisiti obbligatori (29/29)
\item il  90\% dei requisiti desiderabili (8/9)
\item il 	50\% dei requisiti facoltativi (3/6)
\end{itemize}