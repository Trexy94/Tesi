% !TEX encoding = UTF-8
% !TEX TS-program = pdflatex
% !TEX root = ../tesi.tex
% !TEX spellcheck = it-IT

%**************************************************************
% Sommario
%**************************************************************
\cleardoublepage
\phantomsection
\pdfbookmark{Sommario}{Sommario}
\begingroup
\let\clearpage\relax
\let\cleardoublepage\relax
\let\cleardoublepage\relax

\chapter*{Sommario}

Scopo di questa tesi di laurea è esporre il lavoro svolto dal laureando Trevisan Davide
durante lo stage di trecentoventi ore presso l’azienda Ennova Research SRL con sede in
Venezia-Mestre.
Il progetto di stage si è incentrato sullo sviluppo di alcune funzionalità connesse
al sistema di gestione della conoscenza che l’azienda utilizza per gestire la documentazione relativa ai progetti assegnati dalle aziende clienti e monitorare l’operato dei dipendenti. Il sistema si basa sulla piattaforma Alfresco, un noto e molto utilizzato KMS, la quale offre una SDK lanciabile in maniera autonoma tramite Maven che consente lo sviluppo di moduli per personalizzare e ottimizzare la piattaforma a proprio piacimento. Nella realizzazione del progetto è stato possibile, pertanto, approfondire le potenzialità, i difetti e le caratteristiche della piattaforma Alfresco nonché le fasi che hanno portato alla realizzazione di moduli per aggiungere ad Alfresco nuove funzionalità e una maggiore customizzazione del suo aspetto.
Tutte le fasi, le problematiche e quanto prodotto durante il progetto sarà accuratamente esposto nei capitoli che compongono la presente tesi.
%\vfill
%
%\selectlanguage{english}
%\pdfbookmark{Abstract}{Abstract}
%\chapter*{Abstract}
%
%\selectlanguage{italian}

\endgroup			

\vfill

