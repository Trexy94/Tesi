
%**************************************************************
% Acronimi
%**************************************************************
\selectlanguage{italian}
\renewcommand{\acronymname}{Acronimi e abbreviazioni}
\newacronym[description={\glslink{API}{Application Program Interface}}]
    {api}{API}{Application Program Interface}
\newacronym[description={\glslink{KMS}{Knowledge Managment System}}]
    {kms}{KMS}{Knowledge Managment System}
\newacronym[description={\glslink{ICT}{Information and Communication Tecnology}}]
    {ict}{ICT}{Information and Communication Tecnology}
\newacronym[description={\glslink{B2B}{Business-to-business}}]
    {b2b}{B2B}{Business-to-business}		
\newacronym[description={\glslink{B2C}{Business-to-Business}}]
    {b2c}{B2C}{Business-to-Consumer}		
\newacronym[description={\glslink{YUI}{Yahoo! User Interface Library}}]
    {yui}{YUI}{Yahoo! User Interface Library}

%**************************************************************
% Glossario
%**************************************************************
\renewcommand{\glossaryname}{Glossario}

\newglossaryentry{API}
{
    name=\glslink{api}{API},
    text=API,
    sort=api,
    description={in informatica con il termine \emph{Application Programming Interface API} (ing. interfaccia di programmazione di un'applicazione) si indica ogni insieme di procedure disponibili al programmatore, di solito raggruppate a formare un set di strumenti specifici per l'espletamento di un determinato compito all'interno di un certo programma. La finalità è ottenere un'astrazione, di solito tra l'hardware e il programmatore o tra software a basso e quello ad alto livello semplificando così il lavoro di programmazione}
}

%\newglossaryentry{umlg}
%{
    %name=\glslink{uml}{UML},
    %text=UML,
    %sort=uml,
    %description={in ingegneria del software \emph{UML, Unified Modeling Language} (ing. linguaggio di modellazione unificato) è un linguaggio di modellazione e specifica basato sul paradigma object-oriented. L'\emph{UML} svolge un'importantissima funzione di ``lingua franca'' nella comunità della progettazione e programmazione a oggetti. Gran parte della letteratura di settore usa tale linguaggio per descrivere soluzioni analitiche e progettuali in modo sintetico e comprensibile a un vasto pubblico}
%}
\newglossaryentry{KMS}{
    name=\glslink{kms}{KMS},
    text=KMS,
    sort=uml,
    description={Knowledge management system:\\
		I Knowledge management system sono sistemi software che supportano le fasi del ciclo dell'informazione e la comunicazione all'interno di una comunità di pratica (ad esempio un'azienda) o di apprendimento (ad esempio una classe "virtuale") anche disperse nello spazio. Dovrebbero assistere le persone ad esplicitare la conoscenza tacita, a reperirla, a condividerla, supportando in particolare le seguenti funzioni:
		\begin{itemize}
			\item Cattura delle competenze collettive
\item Controllo per realizzare obiettivi comuni
\item Integrazione delle conoscenze frammentate
		\end{itemize}
}
}
\newglossaryentry{ICT}{
    name=\glslink{ict}{ICT},
    text=ICT,
    sort=ict,
    description={Information and communication tecnology:\\ Le tecnologie dell'informazione e della comunicazione (in inglese Information and Communications Technology, in acronimo ICT), sono l'insieme dei metodi e delle tecnologie che realizzano i sistemi di trasmissione, ricezione ed elaborazione di informazioni (tecnologie digitali comprese)}
		}
\newglossaryentry{B2C}{
   name=\glslink{b2c}{B2C},
    text=B2C,
    sort=b2c,
    description={Con Business to Consumer, spesso abbreviato in \emph{B2C}, si indicano le relazioni che un'impresa commerciale detiene con i suoi clienti per le attività di vendita e/o di assistenza}
		}
\newglossaryentry{B2B}{
    name=\glslink{b2b}{B2B},
    text=B2B,
    sort=b2b,
    description={Business-to-business, spesso indicato con l'acronimo \emph{B2B}, in italiano commercio interaziendale, è una locuzione utilizzata per descrivere le transazioni commerciali elettroniche tra imprese, distinguendole da quelle che intercorrono tra le imprese e altri gruppi, come quelle tra una impresa e il governo}}
\newglossaryentry{YUI}{
    name=\glslink{yui}{YUI},
    text=YUI,
    sort=yui,
    description={YUI è una libreria open-source JavaScript per scrivere applicazioni web interattive usando tecniche come Ajax, DHTML, e scripting DOM. YUI include anche molte risorse CSS nel suo nucleo. È disponibile con licenza BSD.Lo sviluppo su YUI è iniziato nel 2005 e nell'estate di quell'anno iniziò il suo utilizzo sulle pagine di Yahoo!.È stato rilasciato per uso pubblico nel 2006.
Nel settembre 2009, Yahoo! ha rilasciato YUI 3, una versione ricostruita da zero apprendendo dagli errori commessi con YUI 2. Tra le caratteristiche più degne di nota vi è l'adozione di JQuery per la selezione degli elementi per il CSS.
Il 29 agosto 2014 è stato annunciato il termine dello sviluppo attivo, dovuto al calo di interesse per le grandi librerie JavaScript, alla evoluzione dello stesso JavaScript e al proliferare di soluzioni server-side. Gli sviluppi futuri si limiteranno alla sistemazione di bug ritenuti critici}}
\newglossaryentry{Activiti Workflow Engine}{
    name=Activiti Workflow Engine,
    text=Activiti Workflow Engine,
    sort=Activiti Workflow Engine,
    description={Activiti è un leggera piattaforma per workflow e BPM (Business Process Management). Il suo cuore è l'engine BPMN2 (Business Process Model and Notation) per Java. È open-source e distribuito attraverso licenza Apache. Activiti funziona in qualsiasi applicazione Java, su un server, su un cluster o nel cloud. Si integra con Spring, è estremamente leggero e basato su concetti semplici
}}